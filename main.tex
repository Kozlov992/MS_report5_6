\documentclass[a4paper]{article}
\input{header}
\begin{document}
\input{title}
\tableofcontents
\addtocontents{toc}{~\hfill\textbf{Страница}\par}
\newpage
\listoffigures
\addtocontents{lof}{~\hfill\textbf{Страница}\par}
\newpage
\listoftables
\addtocontents{lot}{~\hfill\textbf{Страница}\par}
\newpage
\section{Постановка задачи}
\begin{enumerate}
    \item Сгенерировать двумерные выборки размерами $20,\,60,\,100$ для нормального двумерного распределения $N(x,y,0,0,1,1,\rho)$.\\
    Коэффициент корреляции $\rho$ взять равным $0,\,0.5,\,0.9$.\\
    Каждая выборка генерируется $1000$ раз и для неё вычисляются: среднее значение, среднее значение квадрата и дисперсия коэффициентов корреляции Пирсона, Спирмена и квадрантного коэффициента корреляции.\\
    Повторить все вычисления для смеси нормальных распределений:
    \begin{equation*}
        f(x,y)=0.9N(x,y,0,0,1,1,0.9)+0.1N(x,y,0,0,10,10,-0.9).
    \end{equation*}
    Изобразить сгенерированные точки на плоскости и нарисовать эллипс
    равновероятности.
\end{enumerate}
\section{Теория}
\subsection{Двумерное нормальное распределение}
Двумерная случайная величина $(X, Y)$ называется распределенной нормально, если её плотность вероятности определяется формулой
\begin{align}
    N(x,y,\overline{x},\overline{y},\sigma_x,\sigma_y,\rho)&=\frac{1}{2\pi\sigma_x\sigma_y\sqrt{1-\rho^2}}\times\nonumber\\
    &\times\exp\left\{-\frac{1}{2(1-\rho^2)}\left[\frac{\left(x-\overline{x}\right)^2}{\sigma_x^2}-2\rho\frac{(x-\overline{x})(y-\overline{y})}{\sigma_x\sigma_y}+\frac{\left(y-\overline{y}\right)^2}{\sigma_y^2}\right]\right\},
\end{align}
где $\overline{x},\,\overline{y},\sigma_x,\sigma_y$ - математические ожидания и средние квадратические отклонения компонент $X,\,Y$ соответственно, а $\rho\:-$ коэффициент корреляции. 
\subsection{Корреляционный момент и коэффициент корреляции}
\textit{Корреляционный момент} (\textit{ковариация}) двух случайных величин $X, Y$:
\begin{equation}
    K = \cov{(X,Y)}=\mathbf{M}\left[(X-\overline{x})(Y-\overline{y})\right].
\end{equation}
\textit{Коэффициент корреляции} $\rho$ случайных величин $X,Y$:
\begin{equation}
    \rho=\frac{K}{\sigma_x\sigma_y}.
\end{equation}
\subsection{Выборочные коэффициенты корреляции}
\subsubsection{Выборочный коэффициент корреляции Пирсона}
\textit{Выборочный коэффициент корреляции Пирсона}:
\begin{equation}
    r=\frac{\frac{1}{n}\sum_{i=1}^n \left(x_i-\overline{x}\right)\left(y_i-\overline{y}\right)}{\sqrt{\frac{1}{n}\sum_{i=1}^n\left(x_i-\overline{x}\right)^2 \frac{1}{n}\sum_{i=1}^n\left(y_i-\overline{y}\right)^2}}=\frac{K}{s_X s_Y},
\end{equation}
где $K,\,s_X^2,\,s_Y^2\:-$ выборочные ковариация и дисперсии случайных величин $X, Y$.
\subsubsection{Выборочный квадрантный коэффициент корреляции}
\begin{equation}
    r_Q=\frac{(n_1+n_3)-(n_2+n_4)}{n},
\end{equation}
где $n_1,n_2,n_3,n_4\:-$ количества точек с координатами $(x_i,y_i)$, попавшими соответственно в I, II, III и IV квадранты декартовой системы с осями $x^'=x-\med{x},\,y^'=y-\med{y}$ и с центром в точке с координатами $(\med{x},\med{y})$.
\subsubsection{Выборочный коэффициент ранговой корреляции Спирмена}
Обозначим ранги, соотвествующие значениям переменной $X$, через $u$, а ранги, соответствующие значениям переменной $Y$, $-$ через $v$.
\\\\
\textit{Выборочный коэффициент ранговой корреляции Спирмена}:
\begin{equation}
    r_S=\frac{\frac{1}{n}\sum_{i=1}^n \left(u_i-\overline{u}\right)\left(v_i-\overline{v}\right)}{\sqrt{\frac{1}{n}\sum_{i=1}^n\left(u_i-\overline{u}\right)^2 \frac{1}{n}\sum_{i=1}^n\left(v_i-\overline{v}\right)^2}},
\end{equation}
где $\overline{u}=\overline{v}=\frac{1+2+...+n}{n}=\frac{n+1}{2}\,-$ среднее значение рангов.
\subsection{Эллипсы рассеивания}
Уравнение проекции эллипса рассеивания на плоскость $xOy$:
\begin{equation}\label{eq:ellipse}
    \frac{\left(x-\overline{x}\right)^2}{\sigma_x^2}-2\rho\frac{(x-\overline{x})(y-\overline{y})}{\sigma_x\sigma_y}+\frac{\left(y-\overline{y}\right)^2}{\sigma_y^2}=C,\;\;C\,-\,\text{const}.
\end{equation}
Центр эллипса \eqref{eq:ellipse} находится в точке с координатами $(\overline{x},\overline{y})$, оси симметрии эллипса составляют с осью $Ox$ углы, определяемые уравнением
\begin{equation}
    \tan{2\alpha}=\frac{2\rho\sigma_x\sigma_y}{\sigma_x^2-\sigma_y^2}.
\end{equation}
\section{Реализация}
Лабораторная работа выполнена на языках Python и R в средах PyCharm, Jupyter Notebook, R Studio с использованием следующих библиотек:
\begin{itemize}
    \item Python:
    \begin{enumerate}
        \item scipy (генерация выборок)
        \item statsmodels (построение э. ф. р.)
        \item matplotlib, seaborn (визуализация, построение гистограмм и боксплотов)
        \item numpy (вычисление ряда числовых характеристик)
    \end{enumerate}
    \item R:
    \begin{enumerate}
        \item kmlShape (нахождение дискретного расстояния Фреше)
        \item kableExtra (оформление)
\end{enumerate}
\end{itemize}
\section{Результаты}
\section{Обсуждение}
\section*{Примечание}
\end{document}
